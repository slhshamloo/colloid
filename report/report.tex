\documentclass[12pt,a4paper]{article}
\usepackage{algorithm, algpseudocode, amsmath, amssymb, amsthm, bm, csquotes, emf, empheq, geometry, graphicx, hyperref, listings, mhchem, multirow, siunitx, slashbox, subcaption, upgreek}
\usepackage[italicdiff]{physics}
%\usepackage[section]{placeins}
\usepackage[justification=centering]{caption}
\usepackage[column=O]{cellspace}
\hypersetup{colorlinks=true, urlcolor=cyan}

\title{Interaction of Disk Colloidal Particles with Media of Regular Polygons}
\author{Saleh Shamloo Ahmadi, Mohammad Reza Ejtehadi}
%\date{}

\makeatletter
\g@addto@macro\bfseries{\boldmath}
\makeatother

\setlength\cellspacetoplimit{5pt}
\setlength\cellspacebottomlimit{3pt}
\newcommand{\multlinecell}[1]{\begin{tabular}[c]{@{}c@{}}#1\end{tabular}}

\newcommand{\qfrac}[2]{\left(\frac{#1}{#2}\right)}
\newcommand{\fsqrt}[2]{\sqrt{\frac{#1}{#2}}}
\newcommand{\ddfrac}[2]{{\displaystyle\frac{\displaystyle #1}{\displaystyle #2}}}
\newcommand{\pdvc}[3]{\qfrac{\partial #1}{\partial #2}_{#3}}
\newcommand{\dbar}{{d\mkern-7mu\mathchar'26\mkern-2mu}}
\newcommand*{\defeq}{\mathrel{\vcenter{\baselineskip0.5ex \lineskiplimit0pt
			\hbox{\scriptsize.}\hbox{\scriptsize.}}}
	=}

\begin{document}
	\maketitle
	\begin{abstract}
		
	\end{abstract}
	\section{Introduction}
	\section{Model and Methods}
	We study hard disks interacting with hard regular polygons. The particles and the disks interact
	purely via volume exclusion. The embedding of disks in the medium introduces defects in the
	crystal structure of the polygons. We analyze how the radius of one disk, and the distance of
	two identical disks, affect the size and intensity of these defects. We use the defects of the
	medium as an analogue for the energy of the configuration, and hence measure the intensity of
	the interaction of the disks through it.

	Monte Carlo (MC) simulations of $N$ identical polygonal particles with $n$ sides ($n$-gons) of
	diameter $\sigma$ and disks of radius $r$ were performed in the $NPT$ ensemble. The shape of the
	simulation box is a parallelogram (and not a simple rectangle, in order to allow the system to
	equilibrate more easily into the optimal configuration). The MC moves consist of translational
	and rotational moves for single polygons, changing the length of one of the sides of the
	simulation box, and shearing the simulation box. The dimensionless pressure of the system is
	defined as $P^*=P\sigma^2/k_BT$, where $T$ is the temperature of the system and $k_BT = 1$ is
	fixed (The exact value of $k_BT$ only affects the dimensions of pressure).

	We use our own custom Julia code for the simulation. HOOMD-blue offers great capabilities for
	hard-particle MC simulations, but it has limitations regarding 2D systems. In particular, it is
	not possible to introduce external constraints on the systems, like the disks in our
	simulations.

	Systems were initialized close to the dense packing of the polygons, and the disks were embedded
	into the systems by cutting out "holes" in the structure. This was done to avoid making extra
	defects in the system due to the mismatch of different regions in differently oriented dense
	packings, and only count the imperfections due to the presence of the disks.

	The MC moves are proposed in "sweeps", where each sweep consists of one move per particle and
	one box move. in each simulation, $\num{2e6}$ sweeps were performed for equilibriating the
	system (which includes the tuning of the size of each move) and measuremend were performed
	in the next $\num{2e6}$ sweeps.

	To quantify the amount of defects in the system, we calculate the $k$-atic order of particles.
	The smaller this parameter is, the more defects the system has. The $k$-atic order of the $i$th
	particle is given by
	\begin{equation}
		\psi_k(i) = \frac{1}{N_b}\sum_{j=1}^{N_b}e^{ik\theta_{ij}}
	\end{equation}
	where $N_b$ is the number of neighbors, $j$ iterates through the neighbors of the particle, and
	$\theta_{ij}$ is the angle between the orientations of particles $i$ and $j$. The $k$ parameter
	governs the symmetry of the order parameter and typically matches the number of neighbors to be
	found for each particle. So we choose $N_b = k$ and choose $k$ according to the symmetry of the
	dense packing for each system. We also calculate the mean orientation of the system, defined as
	\begin{equation}
		\Phi_k = \frac{1}{N}\abs{\sum_{j=1}^N e^{ik\theta_j}}
	\end{equation}
	to compare to the $k$-atic order and show how only the $k$-atic order can be used as a suitable
	parameter for detecting defects in the system.

	Four sets of simulations were performed:
	\begin{enumerate}
		\item A simulation of only the polygons with no disk, to reproduce the work of Shen et al
		and check the correctness of the simulation.
		\item Systems with one disk in the center, with varying radii.
		\item Systems with two disks on fixed coordinates.
		\item Systems with two disks were the distance between them scales with the side-length
		of the simulation box which aligns with the line passing through the centers of the disks.
	\end{enumerate}
\end{document}
